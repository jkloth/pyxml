\documentclass{howto}

% $Id: xml-howto.tex,v 1.12 2002/05/12 23:33:43 akuchling Exp $

% TODO:
%   Read over and update DOM section
%   Write outline of entire document
%   Explain namespaces somewhere
% XXX not covered: c14n.py, xml.marshal, scripts, namespaces, TREX/schemas,
% XSLT, XPath
% XXX overview of parsers

\newcommand{\element}[1]{\code{#1}}
\newcommand{\attribute}[1]{\code{#1}}

\title{Python/XML HOWTO}

\release{0.7.1}

\author{A.M. Kuchling}
\authoraddress{\email{akuchlin@mems-exchange.org}}

\begin{document}
\maketitle

\begin{abstract}
\noindent
XML is the eXtensible Markup Language, a subset of SGML intended to
allow the creation and processing of application-specific markup
languages.  Python makes an excellent language for processing XML
data.  This document is a tutorial for the Python/XML package.  It
assumes you're already somewhat familiar with the structure and
terminology of XML, though a brief introduction is supplied.
\end{abstract}

\tableofcontents


%==========================================================================
\section{Introduction to XML\label{section-introduction}}

XML, the eXtensible Markup Language, is a simplified dialect of SGML,
the Standardized General Markup Language.  XML is intended to be
reasonably simple to implement and use, and is already being used for
specifying markup languages for various new standards: MathML for
expressing mathematical equations, Synchronized Multimedia Integration
Language for multimedia presentations, and so forth.

SGML and XML represent a document by tagging the document's various
components with their function or meaning.  For example, a book
contains several parts: it has a title, one or more authors, the text
of the book, perhaps a preface or an index, and so forth.  A markup
languge for writing books would therefore have elements indicating
what the contents of the preface are, what the title is, and so forth.
This logical structure should not be confused with the physical
details of how the document is actually printed on paper.  The index
might be printed with narrow margins in a smaller font than the rest
of the book, but markup usually isn't (or shouldn't be, anyway)
concerned with details such as this.  Instead, other software will
translate from the markup language to a typeset format, handling the
presentation details.

This section will provide a brief overview of XML and a few related
standards, but it's far from being complete because making it complete
would require a full-length book and not a short HOWTO. There's no
better way to get a completely accurate (if rather dry) description
than to read the original W3C Recommendations; you can find links to
them in section~\ref{xml-links}, ``Related Links''.  If you already
know what XML is, you can skip the rest of this section.

Later sections of this HOWTO assume that you're familiar with XML
terminology.  Most sections will use XML terms such as \emph{element}
and \emph{attribute}.  Section~\ref{SAX} does not require that you
have experience with any of the various Java SAX implentations.


\subsection{Elements, Attributes and Entities}

A markup language specified using XML looks a lot like HTML; a
document consists of a single \dfn{element}, which contains
sub-elements, which can have further sub-elements inside them.
Elements are indicated by \dfn{tags} in the text.  Tags are always
inside angle brackets \code{<}~\code{>}.  Elements can either contain
content, or they can be empty.

An element can contain content between opening and closing
tags, as in \code{<name>Euryale</name>}, which is a \element{name}
element containing the data \samp{Euryale}. This content may be text
data, other XML elements, or a mixture of both. 

Elements can also be empty, containing nothing, and are represented as
a single tag ended with a slash.  For example, \code{<stop/>} is an
empty \element{stop} element.  Unlike HTML, XML element names are
case-sensitive; \element{stop} and \element{Stop} are two different
elements.

Opening and empty tags can also contain attributes, which specify
values associated with an element.  For example, in the XML text
\code{<name lang='greek'>Herakles</name>}, the \element{name} element
has a \attribute{lang} attribute which has a value of \samp{greek}.
In \code{<name lang='latin'>Hercules</name>}, 
the attribute's value is \samp{latin}.

XML also includes \dfn{entities} as a shorthand for including a
particular character or a longer string.  Entity references always
begin with a \samp{\&} and end with a \samp{;}.  For example, a
particular Unicode character can be written as \code{\&\#4660;} using
its character code in decimal, or as \code{\&\#x1234;} using
hexadecimal.  It's also possible to define your own entities, making
\code{\&title;} expand to ``The Odyssey'', for example.  If you want to
include the \samp{\&} character in XML content, it must be written as
\code{\&amp;}.


\subsection{Well-Formed XML}

A legal XML document must, as a minimum, be \dfn{well-formed}: each
opening tag must have a corresponding closing tag, and tags must nest
properly.  For example, \code{<b><i>text</b></i>} is not well-formed
because the \element{i} element should be enclosed inside the
\element{b} element, but instead the closing \code{</b>} tag is
encountered first.  This example can be made well-formed by swapping
the order of the closing tags, resulting in \code{<b><i>text</i></b>}.

If you've ever written HTML by hand, you may have acquired the habit
of being a bit sloppy about this.  Strictly speaking HTML has exactly
the same rules about nesting tags as XML, but most Web browsers are
very forgiving of errors in HTML.  This is convenient for HTML
authors, but it makes it difficult to write programs to parse HTML
input because the programs have to cope with all sorts of malformed
input.

The authors of the XML specification didn't want XML to fall into the
same trap, because it would make XML processing software much harder
to write.  Therefore, all XML parsers have to be strict and must
report an error if their input isn't well-formed.  The Expat parser
includes an executable program named \program{xmlwf} that parses the
contents of files and reports any well-formedness violations; it's
very handy for checking XML data that's been output from a program or
written by hand.


\subsection{DTDs}

Well-formedness just says that all tags nest properly and that every
opening tag is matched by a closing tag.  It says nothing about the
order of elements or about which elements can be contained inside other
elements.

The following XML, apparently representing a book, is well-formed but
it doesn't match the structure expected for a book:

\begin{verbatim}
<book>
  <index>  ... </index>
  <chapter> ... </chapter>
  <chapter> ... </chapter>
  <abstract>  ... </abstract>
  <chapter> ... </chapter>
  <preface> .... </preface>
</book>
\end{verbatim}

Prefaces don't come at the end of books, the index doesn't belong at
the front, and the abstract doesn't belong in the middle.
Well-formedness alone doesn't provide any way of enforcing that order.
You could write a Python program that took an XML file like this and
checked whether all the parts are in order, but then someone wanting
to understand what documents are legal would have to read your program.

Document Type Definitions, or \dfn{DTDs} for short, are a more concise
way of enforcing ordering and nesting rules. A DTD declares the
element names that are allowed, and how elements can be nested inside
each other.  To take an example from HTML, the \element{LI} element,
representing an entry in a list, can only occur inside certain
elements which represent lists, such as \element{OL} or \element{UL}.
The DTD also specifies the attributes that can be provided for each
element, the default value for each attribute, and whether the
attribute can be omitted.  A \dfn{validating parser} can take a
document and a DTD, and check whether the document is legal according
to the DTD's rules.  (The PyXML package includes a validating parser
called xmlproc.)

DTDs are therefore an example of a \dfn{schema language}, a language
for specifying a set of legal XML documents.  Other applications want
even stricter control over which documents are legal, and there are
therefore stricter schema languages.  XML Schema provides a type
system and a number of basic types, so you can say that the value of
an attribute must be a number or a date.  RELAX NG is another schema
language that provides more power and flexibility than XML Schema, but
is simpler to read and implement.

Note that it's quite possible to get useful work done without using
any schema language at all.  You might decide that just writing
well-formed XML and checking it with a Python program is all you need.
There's no reason to drag in a schema language if it won't be useful.

Let's return to DTDs.  A DTD lists the supported elements, the order
in which elements must occur, and the possible attributes for each
element.  Here's a fragment from an imaginary DTD for writing books:

\begin{verbatim}
<!ELEMENT book (abstract?, preface, chapter*, appendix?)>
<!ELEMENT abstract ...>
<!ELEMENT chapter ...>
<!ATTLIST chapter id    ID    #REQUIRED 
                  title CDATA #IMPLIED>
\end{verbatim}

The first line declares the \element{book} element, and specifies the
elements that can occur inside it and the order in which the
subelements must be provided.  DTDs borrow from regular expression
notation in order to express how elements can be repeated; \samp{?}
means an element must occur 0 or 1 times, \samp{*} is 0 or more times,
and \samp{+} means the element must occur 1 or more times.  For
example, the above declarations imply that the \element{abstract} and
\element{appendix} elements are optional inside a \element{book}
element.  Exactly one \element{preface} element has to be present, and
it can be followed by any number of \element{chapter} elements; having
no chapters at all would be legal.

The \code{ATTLIST} declaration specifies attributes for the
\element{chapter} element.  Chapters can have two attributes,
\attribute{id} and \attribute{title}.  \attribute{title} contains
character data (CDATA) and is optional (that's what \samp{\#IMPLIED}
means, for obscure historical reasons).  \attribute{id} must contain
an ID value, and it's required and not optional.  

A validating parser could take this DTD and a sample document, and
report whether the document is \dfn{valid} according to the rules of
the DTD.  A document is valid if all the elements occur in the right
order, and in the right number of repetitions.


\subsection{Related Links}

For the full details of XML's syntax, the definitive source is the XML
1.0 specification, available on the Web at
\url{http://www.w3.org/TR/xml-spec.html}.  However, like all
specifications it's quite formal and isn't intended to be a friendly
introduction or a tutorial.  The annotated version of the standard, at
\url{http://www.xml.com/xml/pub/a/axml/axmlintro.html}, is quite helpful
in clarifying the specification's intent.  There are also many more
informal tutorials and books available to introduce you to XML at
greater length.

The XML Cover Pages, at \url{http://xml.coverpages.org}, are an
extensive collection of links to XML and SGML resources, including a
news page that's updated every few days.  If you can only remember one
XML-related URL, remember this one.  Cafe con Leche, 
at \url{http://www.ibiblio.org/xml/}, is another good resource.

The xml-dev mailing list is a high-traffic list for implementation and
development; see \url{http://www.xml.org/xml/xmldev.shtml} for
archives and subscription information.  Be warned: Some people might
find the discussion too focused on vague theorizing about information
representation, and not on inventing new standards and tools or
applying existing standards.


%==========================================================================
\section{XML-Related Standards\label{section-standards}}

XML 1.0 is the basic standard, but people have built many, \emph{many}
additional standards and tools on top of XML or to be used with XML.
This section will quickly introduce some of these related
technologies, paying particular attention to those that are supported
by the Python/XML package.

\begin{itemize}

  \item[\textbf{SAX}] 
The Simple API for XML isn't a standard in the formal sense that XML
or ANSI C are.  Rather, SAX is an informal specification originally
designed by David Megginson with input from many people on the xml-dev
mailing list.  SAX defines an event-driven interface for parsing XML.
To use SAX, you must create Python class instances which implement a
specified interface, and the parser will then call various methods on
those objects.  See section~\ref{section-SAX}.

  \item[\textbf{DOM}] 
The Document Object Model specifies a tree-based representation for an
XML document, as opposed to the event-driven processing provided by
SAX.  See section~\ref{section-DOM}.

  \item[\textbf{Namespaces}] One XML document can refer to elements
  from more than one DTD.  (Such documents can no longer be validated
  using DTDs, though other schema languages such as RELAX NG can
  handle namespaces.) For example, a document might contain both some
  text and a diagram.  The text might be represented using some
  elements from the HTML DTD, and the diagram might use elements from
  the Scalable Vector Graphics DTD.  All the relevant modules in the
  PyXML module can be used for namespace-aware processing.

  \item[\textbf{XPath and XPointer}]
  XPath is a language for referring to parts of an XML document.  With
XPath you can refer to paragraph number N, or ``all paragraphs of
class \samp{warning}'', or all chapters that have one or more
subsections.
XPointer defines a way to use XPath declarations as the fragment
identifier in a URL to point at a part of an XML document.
See section~\ref{section-XPath}.

  \item[\textbf{XSLT}] 
XSLT is a general tool for transforming one XML document
into another document, specifying the transformation using 
another XML document called a \dfn{stylesheet}.
See section~\ref{section-XSLT}.

  \item[\textbf{RDF}] 
  The Resource Description Format is for describing metadata about 
  other resources.  The PyXML package doesn't contain any support
  for RDF, but a Python library called Redfoot
  (\url{http://redfoot.sf.net}) is available.

\end{itemize}


%==========================================================================
\section{Installing the XML Toolkit\label{section-install}}

Releases are available from
\url{http://sourceforge.net/projects/pyxml/}.
Windows users should download the appropriate precompiled version.
Linux users can either download an RPM, or install from source.  Users
on other platfoms have no choice but to install from source.

To compile from source on a \UNIX{} platform, simply perform the
following steps.

\begin{enumerate}

\item Download the latest version of the source distribution from
\url{http://sourceforge.net/projects/pyxml}.  Unpack it with the
following command.

\begin{verbatim}
gzip -dc xml-package.tgz | tar -xvf -
\end{verbatim}

\item Run \code{python setup.py install}.  In order to run this,
you'll need to have a C compiler installed, and it should be the same
one that was used to build your Python installation. On a Unix system,
this operation may require superuser permissions. \code{setup.py}
supports a number of different commands and options; invoke
\code{setup.py} without any arguments to see a help message.

\end{enumerate}

If you have difficulty installing this software, send a problem report
to the XML-SIG mailing list describing the problem, or submit a bug report
at \url{http://sourceforget.net/projects/pyxml}.

There are various demonstration programs in the \file{demo/} directory
of the Python/XML source distribution.  You may wish to look at them
to get an idea of what's possible with the XML tools, and as a source
of example code.


\subsection{Related Links}

The Python/XML Topic Guide, at
\url{http://pyxml.sourceforge.net/topics/} is the starting point for
Python-related XML topics; it links to software, mailing lists,
documentation, etc.


%==========================================================================
\section{Package Overview}

The PyXML package contains over 200 individual modules, some intended
for public use and some not.  Many of these modules often perform
similar tasks, making it difficult to figure out which is the right
one to use in any given situation, and this can make it confusing.
Here's a list of the 30-odd packages and modules that are considered
public, along with brief descriptions to help you choose the right
one.

\begin{itemize}

\item[\module{xml.dom}]
  The Python DOM interface.  The full interface
  support DOM Levels 1 and 2.  \module{xml.dom} contains
  the implementation for DOM trees built from XML documents.
  (This implementation is called 4DOM, and was written by Fourthought LLC.)
  
\item[\module{dom.html}]
  DOM trees built from HTML documents are also supported.

\item[\module{dom.javadom}]
  An adaptor for using Java DOM implementations with Jython.

\item[\module{dom.minidom}]
  A lightweight DOM implementation that's also included in the Python
  standard library.  

\item[\module{dom.minitraversal}]
  Offers traversal and ranges on top of
  \module{xml.dom.minidom}, using the 4DOM traversal implementation.

\item[\module{dom.pulldom}]
  Provides a stream of DOM elements.  This module can make it easy 
  to write certain types of DTD-specific processing code.

\item[\module{dom.ext}]
  Various DOM-related extensions for pretty-printing DOM trees as XML or XHTML.

\item[\module{dom.ext.Dom2Sax}]
  A parser to generate SAX events from a DOM tree.

\item[\module{dom.ext.Visitor}]
  Classes for walking over a DOM tree.

\item[\module{dom.ext.c14n}]
  Takes a DOM tree and outputs a text stream containing the
  Canonical XML representation of the document.

\item[\module{dom.ext.reader}]
  Classes for building DOM trees from various input sources:
  SAX1 and SAX2 parsers, \module{htmllib}, and directly using Expat.

\item[\module{xml.marshal.generic}]
  Marshals simple Python data types into an XML format.
  The marshaller class can be subclassed in order to implement
  marshalling into a different XML DTD.

\item[\module{xml.marshal.wddx}]
  Marshals Python objects into WDDX.  (This module is built on top 
  of the preceding generic module.)

\item[\module{xml.ns}]
  Contains constants for the namespace URIs for various XML-related standards.

\item[\module{xml.parsers.sgmllib}]
  A version of the \module{sgmllib} module that's part of the standard 
  Python library, rewritten to run on top of the \module{sgmlop}
  accelerator module.

\item[\module{xml.parsers.xmlproc}]
  A validating XML parser.  Usually you'll want to use xmlproc via SAX or
  some other higher-level interface.

\item[\module{xml.sax}]
  SAX1 and SAX2 support for Python.

\item[\module{sax.drivers}]
  SAX1 drivers for various parsers: \module{htmllib}, 
  LT, Expat, \module{sgmllib}, \module{xmllib}, xmlproc, 
  and XML-Toolkit.

\item[\module{sax.drivers2}]
  SAX2 drivers for various parsers: \module{htmllib}, Java SAX parsers
  (for Jython), Expat, \module{sgmllib}, xmlproc.

\item[\module{sax.handler}] 
  Contains the core SAX2 handler classes \class{ContentHandler},
  \class{DTDHandler}, \class{EntityResolver}, and
  \class{ErrorHandler}.  Also contains symbolic names for the various
  SAX2 features and properties.


\item[\module{sax.sax2exts}]
  SAX2 extensions.  This contains various factory classes that create
  parser objects, and is how SAX2 parsers are used.

\item[\module{sax.saxlib}]
  Contains two SAX2 handler classes, \class{DeclHandler} and
  \class{LexicalHandler}, and the \class{XMLFilter} interface.  
  Also contains the deprecated SAX1 handler classes.

\item[\module{sax.saxutils}]
  Various utility classes, such as \class{DefaultHandler}, a default
  base class for SAX2 handlers, \class{ErrorPrinter} and
  \class{ErrorRaiser}, two default error handlers, and
  \class{XMLGenerator}, which generates XML output from a SAX2 event stream.
  
\item[\module{sax.xmlreader}]
  Contains the \class{XMLReader}, the base interface for implementing 
  SAX2 parsers.

\item[\module{xml.schema.trex}]
  A Python implementation of TREX, a schema language.

\item[\module{xml.utils.characters}]
  Contains the legal XML character ranges as specified in the XML 1.0
  Recommendation, and regular expressions that match various
  XML tokens.
  
\item[\module{xml.utils.iso8601}]
  Parses ISO-8601 date/time specifiers, which look like 
  \samp{2002-05-09T20:40Z}.
  
\item[\module{xml.utils.qp_xml}]
  A simple tree-based XML parsing interface.  

\item[\module{xml.xpath}]
  An XPath parser and evaluator.
  (This implementation is called 4XPath, and was written by Fourthought LLC.)

\item[\module{xml.xslt}] 
   An implementation of the XSLT transformation language.
  (This implementation is called 4XSLT, and was written by Fourthought LLC.)

\end{itemize}


%==========================================================================
\section{SAX: The Simple API for XML\label{section-SAX}}


This HOWTO describes version 2 of SAX (also referred to as SAX2).
Support is still present for SAX version 1, which is now only of
historical interest; SAX1 will not be documented here.

SAX is most suitable for purposes where you want to read through an
entire XML document from beginning to end, and perform some
computation such as building a data structure or summarizing the
contained information (computing an average value of a certain
element, for example).  SAX is not very convenient if you want to
modify the document structure by changing how elements are nested,
though it would be straightforward to write a SAX program that simply
changed element contents or attributes.  For example, you wouldn't
want to re-order chapters in a book using SAX, but you might want to
extract the contents of all \element{name} elements with the attribute
\attribute{lang} set to 'greek'.

One advantage of SAX is speed and simplicity.  Let's say
you've defined a complicated DTD for listing comic books, and you wish
to scan through your collection and list everything written by Neil
Gaiman.  For this specialized task, there's no need to expend effort
examining elements for artists and editors and colourists, because
they're irrelevant to the search.  You can therefore write a class
instance which ignores all elements that aren't \element{writer}.

Another advantage of SAX is that you don't have the whole document
resident in memory at any one time, which matters if you are
processing really huge documents.  

SAX defines 4 basic interfaces. A SAX-compliant XML parser can be
passed any objects that support these interfaces, and will call
various methods as data is processed.  Your task, therefore, is to
implement those interfaces that are relevant to your application.

The SAX interfaces are:

\begin{tableii}{c|p{4in}}{code}{Interface}{Purpose}

\lineii{ContentHandler}{Called for general document events.  This
interface is the heart of SAX; its methods are called for the start of
the document, the start and end of elements, and for the characters of
data contained inside elements.
}

\lineii{DTDHandler}{Called to handle DTD events required for basic
parsing.  This means notation declarations (XML spec section 4.7) and
unparsed entity declarations (XML spec section 4).
}

\lineii{EntityResolver}{Called to resolve references to external
entities.  If your documents will have no external entity references,
you don't need to implement this interface.}

\lineii{ErrorHandler}{Called for error handling.  The parser will call
methods from this interface to report all warnings and errors.}

\end{tableii}

Python doesn't support the concept of interfaces, so the interfaces
listed above are implemented as Python classes.  The default method
implementations are defined to do nothing---the method body is just a
Python \code{pass} statement---so usually you can simply ignore methods
that aren't relevant to your application. 

Pseudo-code for using SAX looks something like this:
\begin{verbatim}
# Define your specialized handler classes
from xml.sax import ContentHandler, ...
class docHandler(ContentHandler):
    ...

# Create an instance of the handler classes
dh = docHandler()

# Create an XML parser
parser = ...

# Tell the parser to use your handler instance
parser.setContentHandler(dh)

# Parse the file; your handler's methods will get called
parser.parse(sys.stdin)
\end{verbatim}

\subsection{Starting Out}

Let's follow the earlier example of a comic book collection, using a
simple DTD-less format. Here's a sample document for a collection
consisting of a single issue:

\begin{verbatim}
<collection>
  <comic title="Sandman" number='62'>
    <writer>Neil Gaiman</writer>
    <penciller pages='1-9,18-24'>Glyn Dillon</penciller>
    <penciller pages="10-17">Charles Vess</penciller>
  </comic>
</collection>
\end{verbatim}

An XML document must have a single root element; this is the
\samp{collection} element.  It has one child \element{comic} element
for each issue; the book's title and number are given as attributes of
the \element{comic} element.  The \element{comic} element can in turn
contain several other elements such as \element{writer} and
\element{penciller} listing the writer and artists responsible for the
issue.  There may be several artists or writers for a single issue.

Let's start off with something simple: a document handler named
\class{FindIssue} that reports whether a given issue is in the
collection.

\begin{verbatim}
from xml.sax import saxutils

class FindIssue(saxutils.DefaultHandler):
    def __init__(self, title, number):
        self.search_title, self.search_number = title, number
\end{verbatim}

The \class{DefaultHandler} class inherits from all four interfaces:
\class{ContentHandler}, \class{DTDHandler}, \class{EntityResolver},
and \class{ErrorHandler}.  This is what you should use if you want to
just write a single class that wraps up all the logic for your
parsing.  You could also subclass each interface individually and
implement separate classes for each purpose.  Neither of the two
approaches is always ``better'' than the other; mostly it's a matter
of taste.

Since this class is doing a search, an instance needs to know what
it's searching for.  The desired title and issue number are passed to
the \class{FindIssue} constructor, and stored as part of the instance.

Now let's override some of the parsing methods.
This simple search only requires looking at the attributes of a given
element, so only the \method{startElement} method is relevant.

\begin{verbatim}
    def startElement(self, name, attrs):
        # If it's not a comic element, ignore it
        if name != 'comic': return

        # Look for the title and number attributes (see text)
        title = attrs.get('title', None)
        number = attrs.get('number', None)
        if (title == self.search_title and 
	    number == self.search_number):
            print title, '#' + str(number), 'found'
\end{verbatim}

The \method{startElement()} method is passed a string giving the name
of the element, and an instance containing the element's attributes.
Attributes are accessed using 
methods from the \class{AttributeList} interface, which
includes most of the semantics of Python dictionaries.  

To summarize, the \method{startElement()} method looks for
\element{comic} elements and compares the specified \attribute{title}
and \attribute{number} attributes to the search values.  If they
match, a message is printed out.

\method{startElement()} is called for every single element in the
document.  If you added \code{print 'Starting element:', name} to the
top of  \method{startElement()}, you would get the following output.

\begin{verbatim}
Starting element: collection
Starting element: comic
Starting element: writer
Starting element: penciller
Starting element: penciller
\end{verbatim}

To actually use the class, we need top-level code that creates
instances of a parser and of \class{FindIssue}, associates the parser
and the handler, and then calls a parser method to process the input.

\begin{verbatim}
from xml.sax import make_parser
from xml.sax.handler import feature_namespaces

if __name__ == '__main__':
    # Create a parser
    parser = make_parser()

    # Tell the parser we are not interested in XML namespaces
    parser.setFeature(feature_namespaces, 0)

    # Create the handler
    dh = FindIssue('Sandman', '62')

    # Tell the parser to use our handler
    parser.setContentHandler(dh)

    # Parse the input
    parser.parse(file)
\end{verbatim}

The \function{make_parser} class can automate the job of creating
parsers.  There are already several XML parsers available to Python,
and more might be added in future.  \file{xmllib.py} is included as
part of the Python standard library, so it's always available, but
it's also not particularly fast.  A faster version of \file{xmllib.py}
is included in \module{xml.parsers}.  The \module{xml.parsers.expat}
module is faster still, so it's obviously a preferred choice if it's
available.  \function{make_parser} determines which parsers are
available and chooses the fastest one, so you don't have to know what
the different parsers are, or how they differ. (You can also tell
\function{make_parser} to try a list of parsers, if you want to use a
specific one).

Once you've created a parser instance, calling the
\method{setContentHandler()} method tells the parser what to use as
the content handler.  There are similar methods for setting the other
handlers: \method{setDTDHandler()}, \method{setEntityResolver()}, and
\method{setErrorHandler()}.

If you run the above code with the sample XML document, it'll print
\code{Sandman \#62 found.}  

\subsection{Error Handling}

Now, try running the above code with this file as input:
\begin{verbatim}
<collection>
  &foo;
  <comic title="Sandman" number='62'>
</collection>
\end{verbatim}

The \code{\&foo;} entity is unknown, and the \element{comic} element
isn't closed (if it was empty, there would be a \samp{/} before the
closing \samp{>}. As a result, you get a
\exception{SAXParseException}, e.g.

\begin{verbatim}
xml.sax._exceptions<.SAXParseException: undefined entity at None:2:2
\end{verbatim}

The default code for the \class{ErrorHandler} interface automatically
raises an exception for any error; if that is what you want, you don't
need to implement an error handler class at all.  Otherwise, you can
provide your own version of the \class{ErrorHandler} interface, at
minimum overriding the \method{error()} and \method{fatalError()}
methods.  The minimal implementation for each method can be a single
line.  The methods in the \class{ErrorHandler}
interface---\method{warning()}, \method{error()}, and
\method{fatalError()}---are all passed a single argument, an exception
instance.  The exception will always be a subclass of
\exception{SAXException}, and calling \code{str()} on it will produce
a readable error message explaining the problem.

For example, if you just want to continue running if a recoverable
error occurs, simply define the \method{error()} method to print the
exception it's passed:

\begin{verbatim}
    def error(self, exception):
        import sys
        sys.stderr.write("\%s\n" \% exception)
\end{verbatim}

With this definition, non-fatal errors will result in an error message,
whereas fatal errors will continue to produce a traceback.

\subsection{Searching Element Content}

Let's tackle a slightly more complicated task: printing out all issues
written by a certain author.  This now requires looking at element
content, because the writer's name is inside a \element{writer}
element: \code{<writer>Peter Milligan</writer>}.

The search will be performed using the following algorithm:

\begin{enumerate}
\item 
The \method{startElement} method will be more complicated.  For
\element{comic} elements, the handler has to save the title and
number, in case this comic is later found to match the search
criterion.  For \element{writer} elements, it sets a
\code{inWriterContent} flag to true, and sets a \code{writerName}
attribute to the empty string.

\item Characters outside of XML tags must be processed.  When
\code{inWriterContent} is true, these characters must be added to the
\code{writerName} string.

\item When the \element{writer} element is finished, we've now
collected all of the element's content in the \code{writerName}
attribute, so we can check if the name matches the one we're searching 
for, and if so, print the information about this comic.  We must also
set \code{inWriterContent} back to false.
\end{enumerate}

Here's the first part of the code; this implements step 1.

\begin{verbatim}
from xml.sax import ContentHandler
import string

def normalize_whitespace(text):
    "Remove redundant whitespace from a string"
    return ' '.join(text.split())

class FindWriter(ContentHandler):
    def __init__(self, search_name):
        # Save the name we're looking for
        self.search_name = normalize_whitespace(search_name)

        # Initialize the flag to false
        self.inWriterContent = 0

    def startElement(self, name, attrs):
        # If it's a comic element, save the title and issue
        if name == 'comic':
            title = normalize_whitespace(attrs.get('title', ""))
            number = normalize_whitespace(attrs.get('number', ""))
            self.this_title = title
            self.this_number = number

        # If it's the start of a writer element, set flag
        elif name == 'writer':
            self.inWriterContent = 1
            self.writerName = ""
\end{verbatim}

The \method{startElement()} method has been discussed previously.  Now
we have to look at how the content of elements is processed.  

The \function{normalize_whitespace()} function is important, and
you'll probably use it in your own code.  XML treats whitespace very
flexibly; you can include extra spaces or newlines wherever you like.
This means that you must normalize the whitespace before comparing
attribute values or element content; otherwise the comparison might
produce an incorrect result due to the content of two elements having
different amounts of whitespace.

\begin{verbatim}
    def characters(self, ch):
        if self.inWriterContent:
            self.writerName = self.writerName + ch
\end{verbatim}

The \method{characters()} method is called for characters that aren't
inside XML tags.  \var{ch} is a string of characters. It is not
necessarily a byte string; parsers may also provide a buffer object
that is a slice of the full document, or they may pass Unicode
objects.

You also shouldn't assume that all the characters are passed in a
single function call.  In the example above, there might be only one
call to \method{characters()} for the string \samp{Peter Milligan}, or
it might call \method{characters()} once for each character.  Another,
more realistic example: if the content contains an entity reference,
as in \samp{Wagner \&amp; Seagle}, the parser might call the method
three times; once for \samp{Wagner\ }, once for \samp{\&}, represented
by the entity reference, and again for \samp{\ Seagle}.

For step 2 of the algorithm, \method{characters()} only has to
check \code{inWriterContent}, and if it's true, add the characters to
the string being built up.

Finally, when the \element{writer} element ends, the entire name has
been collected, so we can compare it to the name we're searching for.

\begin{verbatim}
    def endElement(self, name):
        if name == 'writer':
            self.inWriterContent = 0
            self.writerName = normalize_whitespace(self.writerName)
            if self.search_name == self.writerName:
                print 'Found:', self.this_title, self.this_number
\end{verbatim}

To avoid being confused by differing whitespace, the
\function{normalize_whitespace()} function is called.  This can be
done because we know that leading and trailing whitespace are
insignificant for this application.  

End tags can't have attributes on them, so there's no \var{attrs}
parameter to the \method{endElement()} method.  Empty elements with
attributes, such as \samp{<arc name="Season of Mists"/>}, will result
in a call to \method{startElement()}, followed immediately by a call
to \method{endElement()}.

\subsection{Enabling Namespace Processing}

SAX2 supports XML namespaces.  If namespace processing is active,
parsers won't call \method{startElement()}, but instead will call a
method named \method{startElementNS()}. The default of this
setting varies from parser to parser, so you should always set it to a
safe value (unless your handler supports both namespace-aware and
-unaware processing).

For example, our \class{FindIssue} content handler described in
previous section doesn't implement the namespace-aware methods, so we
should request that namespace processing is deactivated before
beginning to parse XML:

\begin{verbatim}
from xml.sax import make_parser
from xml.sax.handler import feature_namespaces

# Create a parser
parser = make_parser()

# Disable namespace processing
parser.setFeature(feature_namespaces, 0)
\end{verbatim}

The second argument to \method{setFeature()} is the desired state of
the feature, mostly commonly a Boolean.  You would call
\code{parser.setFeature(feature_namespaces, 1)} to enable namespace
processing.

Namespaces in XML work by first defining a namespace prefix that maps
to a given URI specified by the relevant DTD, and then using that
prefix to mark elements and attributes that come from that DTD.  For
example, the XLink specification says that the namaspace URI is 
\samp{http://www.w3.org/1999/xlink}.  The following XML snippet
includes some XLink attributes:

\begin{verbatim}
<root xmlns:xlink="http://www.w3.org/1999/xlink">
  <elem xlink:href="http://www.python.org" />
</root>
\end{verbatim}

The \attribute{xmlns:xlink} attribute on the \element{root} element
declares that the prefix \samp{xlink} maps to the given URL.  The
\element{elem} element therefore has one attribute named
\attribute{href} that comes from the XLink namespace.  Namespace-aware
methods expect \code{(\var{URI}, \var{name})} tuples instead of just
element and attribute names; instead of \samp{xlink:href}, they would
receive \code{('http://www.w3.org/1999/xlink', 'href')}.

Note that the actual value of the prefix is immaterial, and software
shouldn't make assumptions about it.  The XML document would have
exactly the same meaning if the root element said
\samp{xmlns:pref1="http://..."} and the attribute name was given as
\samp{pref1:href}.

If namespace processing is turned on, you would have to write
\method{startElementNS()} and \method{endElementNS()} methods that
looked like this:

\begin{verbatim}
    def startElementNS(self, (uri, localname), qname, attrs):
        ...

    def endElementNS(self, (uri, localname, qname):
        ...
\end{verbatim}

The first argument is a 2-tuple containing the URI and the name of the
element within that namespace.  \var{qname} is a string containing the
original qualified name of the element, such as \samp{xlink:a}, and
\var{attrs} is a dictionary of attributes.  The keys of this
dictionary will be \code{(\var{URI}, \var{attribute_name})} pairs.  If
no namespace is specified for an element or attribute, the URI will
given given as \code{None}.


\subsection{Related Links}

The SAX home page is \url{http://www.megginson.com/SAX/}.  This has
the most recent copy of the specification, and lists SAX
implementations for various languages and platforms.  At the moment
it's somewhat Java-centric, though.


%==========================================================================
\section{DOM: The Document Object Model}
\label{section-DOM}

With SAX you write a class which then gets the entire document poured
through it as a sequence of method calls.  An alternative approach is
that taken by the Document Object Model, or DOM, which turns an XML
document into a tree that's fully resident in memory.  

A top-level \class{Document} instance is the root of the tree, and has
a single child which is the top-level \class{Element} instance; this
\class{Element} has child nodes representing the content and any
sub-elements, which may in turn have further children and so forth.
There are different classes for everything that can be found in an XML
document, so in addition to the \class{Element} class, there are also
classes such as \class{Text}, \class{Comment}, \class{CDATASection},
\class{EntityReference}, and so on.  Nodes have methods for
accessing the parent and child nodes, accessing element and attribute
values, insert and delete nodes, and converting the tree back into XML.

The DOM is often useful for modifying XML documents, because you can
create a DOM tree, modify it by adding new nodes and moving subtrees
around, and then produce a new XML document as output.  On the other
hand, while the DOM doesn't require that the entire tree be resident
in memory at one time, the Python DOM implementation currently keeps
the whole tree in RAM.  This means you may not have enough memory to
process very large documents as a DOM tree.  A SAX handler, on the
other hand, can potentially churn through amounts of data far larger
than the available RAM.

This HOWTO can't be a complete introduction to the Document Object
Model, because there are lots of interfaces and lots of
methods. Luckily, the DOM Recommendation is quite readable, so I'd
recommend that you read it to get a complete picture of the available
interfaces.  This section will only be a partial overview.


\subsection{Getting A DOM Tree}

The easiest way to get a DOM tree is to have it built for you. PyXML
offers two alternative implementations of the DOM,
\module{xml.dom.minidom} and \code{4DOM}. \module{xml.dom.minidom} is
included in Python 2. It is a minimal implementation, which means it
does not provide all interfaces and operations required by the DOM
standard.  \code{4DOM} (XXX reference) is a complete implementation of
DOM Level 2 (which is currently work in progress), so we will use that
in the examples.

The \module{xml.dom.ext.reader} package contains a number of classes
that build a DOM tree from various input sources.  One of the modules
in the \module{xml.dom} package is named \module{Sax2}, and contains a
\class{Reader} class that builds a DOM tree from a series of SAX2
events.  \class{Reader} instances provide a \method{fromStream()}
method that constructs a DOM tree from an input stream; the input can
be a file-like object or a string.  In the second case, it will be
assumed to be a URL and will be opened with the \module{urllib}
module.

\begin{verbatim}
import sys
from xml.dom.ext.reader import Sax2

# create Reader object
reader = Sax2.Reader()

# parse the document
doc = reader.fromStream(sys.stdin)
\end{verbatim}

\method{fromStream()} returns the root of a DOM tree constructed from
the input XML document.


\subsection{Printing The Tree}

We'll use a single example document throughout this section.  Here's
the sample:

\begin{verbatim}
<?xml version="1.0" encoding="iso-8859-1"?>
<xbel>  
  <?processing instruction?>
  <desc>No description</desc>
  <folder>
    <title>XML bookmarks</title>
    <bookmark href="http://www.python.org/sigs/xml-sig/" >
      <title>SIG for XML Processing in Python</title>
    </bookmark>
  </folder>
</xbel>
\end{verbatim}

Converted to a DOM tree, this document could produce the following
tree.   
% XXX what did this output come from?

\begin{verbatim}
Element xbel None
   Text #text '  \012  '
   ProcessingInstruction processing 'instruction'
   Text #text '\012  '
   Element desc None
      Text #text 'No description'
   Text #text '\012  '
   Element folder None
      Text #text '\012    '
      Element title None
         Text #text 'XML bookmarks'
      Text #text '\012    '
      Element bookmark None
         Text #text '\012      '
         Element title None
            Text #text 'SIG for XML Processing in Python'
         Text #text '\012    '
      Text #text '\012  '
   Text #text '\012'
\end{verbatim}

This isn't the only possible tree, because different parsers may
differ in how they generate \class{Text} nodes; any of the
\class{Text} nodes in the above tree might be split into multiple
nodes.  

A DOM tree can be converted back to XML by using the
\function{Print(\var{doc}, \var{stream})} or
\function{PrettyPrint(\var{doc}, \var{stream})} functions in
\module{xml.dom.ext}.  If \var{stream} isn't provided, the resulting
XML will be printed to standard output.  \function{Print()} will
simply render the DOM tree without any changes, while
\function{PrettyPrint()} will add or remove whitespace in order to
nicely indent the resulting XML.


\subsection{Manipulating the Tree}

We'll start by considering the basic \class{Node} class.  All the
other DOM nodes---\class{Document}, \class{Element}, \class{Text},
and so forth---are subclasses of \class{Node}.  It's possible to
perform many tasks using just the interface provided by \class{Node}.

First, there are the attributes provided by all \class{Node} instances:

\begin{tableii}{|l|l|}{member}{Attribute}{Meaning}
  \lineii{nodeType}{Integer constant giving the type of this node: 
\constant{ELEMENT_NODE}, \constant{TEXT_NODE}, etc.}
  \lineii{nodeName}{Name of this node.  For some types of node, such
as \class{Element}s, the name is the element name; for others, such as
\class{Text}, the name is a constant value such as \samp{\#text} which
isn't very useful.
}
  \lineii{nodeValue}{Value of this node.  For some types of node, such
as \class{Text} nodes, the value is a string containing
a chunk of textual data; for others, such as
\class{Text}, the value is just \code{None}.}
  \lineii{parentNode}{Parent of this node, or \class{None} if this
node is the root of a tree (usually meaning that it's a
\class{Document} node).}
  \lineii{childNodes}{A possibly empty list containing the children of this node.}
  \lineii{firstChild}{First child of this node, or \code{None} 
if it has no children.}
  \lineii{lastChild}{Last child of this node, or \code{None} 
if it has no children.}
  \lineii{previousSibling}{Preceding child of this node's parent,
or \class{None} if this node has no parent or if the parent has no
preceding children.
}
  \lineii{nextSibling}{Following child of this's node's parent, 
or \class{None} if this node has no parent or if the parent has no
following children.
}
  \lineii{ownerDocument}{Owning document of this node.}
  \lineii{attributes}{A \class{NamedNodeMap} instance 
that behaves mostly like a dictionary, and maps attribute names
to \class{Attribute} instances.}
\end{tableii}

Next, there are the methods.  If a node is already a child of node 1
and is added as a child of node 2, it will automatically be removed
from node 1; nodes always have exactly zero or one parents.

\begin{tableii}{|l|l|}{method}{Method}{Effect}
  \lineii{appendChild(\var{newChild})}{Add \var{newChild} as a child
of this node, adding it to the end of the list of children.
}
  \lineii{removeChild(\var{oldChild})}{Remove \var{oldChild}; its
\member{parentNode} attribute will now return \class{None}.}

  \lineii{replaceChild(\var{newChild}, \var{oldChild}}{Replace
the child \var{oldChild} with \var{newChild}.  
\var{oldChild} must already be a child of the node.}

  \lineii{insertBefore(\var{newChild}, \var{refChild})}{
Add \var{newChild} as  a child of this node, adding it 
before the node \var{refChild}. \var{refChild} 
must already be a child of the node.
}

  \lineii{hasChildNodes()}{Returns true if this node has any children.}
  \lineii{cloneNode(\var{deep})}{Returns a copy of this node.
If \var{deep} is false, the copy will have no children.  If it's true,
then all of the children will also be copied and added as children to 
the returned copy.
}
\end{tableii}

\class{Element} nodes and the \class{Document} node also have a useful
method, \method{getElementsByTagName(\var{tagName})}, that returns a
list of all elements with the given name.  For example, all the
\samp{chapter} elements can be returned by
\code{document.getElementsByTagName('chapter')}.


\subsection{Creating New Nodes}

The base of the entire tree is the \class{Document} node.  Its
\member{documentElement} attribute contains the \class{Element} node
for the root element.  The \class{Document} node may have additional
children, such as \class{ProcessingInstruction} nodes, but the list of
children can include at most one \class{Element} node.

When building a DOM tree from scratch, you'll need to construct new
nodes of various types such as \class{Element} and \class{Text}.  The
\class{Document} node has a bunch of \method{create*()} methods such
as \method{createElement} and \method{createTextNode()}.

For example, here's an example that adds a new child element named
\samp{chapter} to the root element.

\begin{verbatim}
new = document.createElement('chapter')
new.setAttribute('number', '5')
document.documentElement.appendChild(new)
\end{verbatim}


\subsection{Walking Over The Entire Tree}

Once you have a tree, another common task is to traverse it.
\class{Document} instances have a method called
\method{createTreeWalker(\var{root}, \var{whatToShow}, \var{filter},
\var{entityRefExpansion})} that returns an instance of the
\class{TreeWalker} class.

Once you have a \class{TreeWalker} instance, it allows traversing
through the subtree rooted at the \var{root} node.  The
\member{currentNode} attribute contains the current node that's been
reached in this traversal, and can be advanced forward or backward by
calling the \method{nextNode()} and \method{previousNode()} methods.
There are also methods titled \method{parentNode()},
\method{firstChild()}, \method{lastChild()}, and
\method{nextSibling()}, \method{previousSibling()} that return the
appropriate value for the current node.

\var{whattoshow} is a bitmask with bits set for each type of node that
you want to see in the traversal.  Constants are available 
as attributes on the \class{NodeFilter} class.
0 filters out all nodes, \constant{NodeFilter.SHOW_ALL} traverses
every node, and constants such as \constant{SHOW_ELEMENT}
and \constant{SHOW_TEXT} select individual types of node.

\var{filter} is a function that will be passed every traversed node,
and can return \constant{NodeFilter.FILTER_ACCEPT} or
\constant{NodeFilter.FILTER_REJECT} to accept or reject the node.
\var{filter} can be passed as \code{None} in order to accept all
nodes.

% XXX is expandEntityReferences() actually used anywhere?

Here's an example that traverses the entire tree and prints out every
element.

\begin{verbatim}
from xml.dom.NodeFilter import NodeFilter

walker = doc.createTreeWalker(doc.documentElement,
                              NodeFilter.SHOW_ELEMENT, None, 0)

while 1:
    print walker.currentNode.tagName
    next = walker.nextNode()
    if next is None: break
\end{verbatim}

% XXX get a patch in and mention iterators


\subsection{Related Links}

The DOM Level 1 Recommendation is at
\url{http://www.w3.org/TR/1998/REC-DOM-Level-1-19981001/}.  Unlike
most standards, this one is actually pretty readable, particularly if
you're only interested in the Core XML interfaces.

Level 2 of the DOM has also been defined, adding more specialized
features such as support for XML namespaces, events, and ranges.  DOM
Level 3 is still being worked on, and will add yet more features.  The
World Wide Web Consortium's DOM page at \url{http://www.w3.org/DOM/}
has pointers to the specifications and current drafts for these higher
DOM levels.


%==========================================================================
%\section{XPath and XPointer\label{section-XPath}}

% XXX write this


%==========================================================================
\section{XSLT\label{section-XSLT}}

XML documents are often transformed from one format to another.  These
transformations can be minor, such as changing all \element{OL}
elements into \element{UL} elements, or major, such as translating a
DocBook document into HTML so it can be displayed in a Web browser.
You can write a separate Python program to do each transformation as
you need it, and at times that will be the most appropriate option,
but an alternative approach is to use XSL, the Extensible Stylesheet
Language.

XSL is really two standards: XSLT, XSL Transformations; and XSL-FO,
XSL Formatting Objects.  XSLT is used much more often than XSL-FO,
because XSL-FO is intended primarily for rendering XML for printing
onto paper while XSLT is a general tool for transforming one XML document
into another document, and therefore can be used for more diverse
tasks.

To use XSLT, you have to write a \dfn{stylesheet}, which is itself an
XML document written in the XSLT DTD.  The source document is turned
into a tree structure, and the stylesheet specifies the transformation
you want to perform by selecting some elements from the tree and
rearranging them.  

% XXX finish this section

\subsection{Related Links}

The W3C's XSL page is at \url{http://www.w3.org/Style/XSL/}, and links
to the XSLT specifications and to friendlier tutorials.


%==========================================================================
\section{Acknowledgements \label{section-acks}}

The author would like to thank the following people for offering
suggestions, corrections and assistance with various drafts of this
article: Fred~L. Drake, Jr., Martin von L\"owis, Rich Salz.

\end{document}
