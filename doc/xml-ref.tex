% Questions:
% Should xmllib.py be moved into xml.parsers?  What about pyexpat.so?
% Should ErrorPrinter go to standard error, not stdout?

\documentclass{howto}

\newcommand{\element}[1]{\code{#1}}
\newcommand{\attribute}[1]{\code{#1}}

\title{Python/XML Reference Guide}

\release{0.06}

\author{The Python/XML Special Interest Group}
\authoraddress{\email{xml-sig@python.org}\break (edited by \email{akuchling@acm.org})}

\begin{document}
\maketitle

\begin{abstract}
\noindent
XML is the eXtensible Markup Language, a subset of SGML, intended to
allow the creation and processing of application-specific markup
languages.  Python makes an excellent language for processing XML
data.  This document is the reference manual for the Python/XML
package, containing several XML modules.

This is a draft document; 'XXX' in the text indicates that something
has to be filled in later, or rewritten, or verified, or something.  
\end{abstract}

\tableofcontents

\section{xml.arch.xmlarch}

The xmlarch module contains an XML architectural forms processor
   written in Python. It allows you to process XML architectural forms
   using any parser that uses the SAX interfaces. The module allows
   you to process several architectures in one parsing
   pass. Architectural document events for an architecture can even be
   broadcasted to multiple DocumentHandlers. (e.g. you can have 2
   handlers for the RDF architecture, 3 for the XLink architecture and
   perhaps one for the HyTime architecture.)
   
The architecture processor uses the SAX \class{DocumentHandler} interface
   which means that you can register the architecture handler
   (ArchDocHandler) with any SAX 1.0 compliant parser.
   
It currently does not process any meta document type definition
documents (meta-DTDs). When a DTD parser module is available the code
will be modified to use that in order to process meta-DTD information.
   
Please note that validating and well-formed parsers may report
different SAX events when parsing documents.

The \module{xmlarch} module contains six classes:
\class{ArchDocHandler}, \class{Architecture}, \class{ArchParseState},
\class{ArchException}, \class{AttributeParser} and \class{Normalizer}.

\begin{itemize}
     \item \class{ArchDocHandler} is a subclass of the \class{saxlib.DocumentHandler}
       interface. This is the class used for processing an architectural
       document.

     \item \class{Architecture} contains information about an architecture.

     \item \class{ArchParseState} holds information about an architecture's parse
       state when parsing a document.

     \item \class{AttributeParser} parses architecture use declaration PIs (attribute
       strings).

     \item \class{ArchException} holds information about an architectural exception
       thrown by an \class{ArchDocHandler} instance.

     \item \class{Normalizer} is a document handler that outputs "normalized" XML.
\end{itemize}

\subsection{\class{ArchDocHandler} methods}
Still undocumented.
\subsection{\class{Architecture} methods}
Still undocumented.
\subsection{\class{ArchParseState} methods}
Still undocumented.
\subsection{\class{ArchException} methods}
Still undocumented.
\subsection{\class{AttributeParser} methods}
Still undocumented.
\subsection{\class{Normalizer} methods}
Still undocumented.
       
\section{xml.parsers.xmllib}

This is a version of the \module{xmllib} module from Python 1.5,
modified to use the \module{_sgmlop} C extension when it's available.
This produces a significant speedup, amounting to about a factor of 5.
The interface is unchanged from the original \module{xmllib} module;
consult the Python Library Reference documentation for that module.


%\section{{xml.dom}}

%\section{builder}
%\section{core}
%\section{esis_builder}
%\section{html_builder}
%\section{sax_builder}
%\section{transform}
%\section{transformer}
%\section{walker}
%\section{writer}

%\section{{xml.marshal}}

%\section{{xml.sax.drivers}}

%XXX Should all the driver modules be documented, or should they be
%treated as internal modules, whose details will be handled by a parser
%Factory function?

%\section{{xml.sax.drivers.drv_xmllib}}
%\section{{xml.sax.drivers.drv_xmlproc}}
%\section{{xml.sax.drivers.drv_xmlproc_val}}
%\section{{xml.sax.drivers.drv_xmltok}}
%\section{{xml.sax.drivers.drv_xmltoolkit}}

\section{{xml.sax.saxexts}}

\begin{funcdesc}{make_parser}{\optional{parser}}
A utility function that returns a \class{Parser} object for a
non-validating XML parser.  If \var{parser} is specified, it must be a 
parser name; otherwise, a list of available parsers is checked and the
fastest one chosen.
\end{funcdesc}

\begin{datadesc}{HTMLParserFactory}
An instance of the \class{ParserFactory} class that's already been
prepared with a list of HTML parsers.  Simply call its
\method{make_parser()} method to get a \class{Parser} object. 
\end{datadesc}

\begin{classdesc}{ParserFactory}{}
A general class to be used by applications for creating parsers on
foreign systems where the list of installed parsers is unknown.
\end{classdesc}

\begin{datadesc}{SGMLParserFactory}
An instance of the \class{ParserFactory} class that's already been
prepared with a list of SGML parsers.  Simply call its
\method{make_parser()} method to get a parser object. 
\end{datadesc}

\begin{datadesc}{XMLParserFactory}
An instance of the \class{ParserFactory} class that's already been
prepared with a list of nonvalidating XML parsers.  Simply call its
\method{make_parser()} method to get a parser object. 
\end{datadesc}

\begin{datadesc}{XMLValParserFactory}
An instance of the \class{ParserFactory} class that's already been
prepared with a list of validating XML parsers.  Simply call its
\method{make_parser()} method to get a parser object. 
\end{datadesc}

\begin{classdesc}{ExtendedParser}{}
This class is an experimental extended parser interface, that offers
additional functionality that may be useful.  However, it's not
specified by the SAX specification.
\end{classdesc}

\subsection{\class{ExtendedParser} methods}

\begin{methoddesc}{close}{}
Called after the last call to feed, when there are no more data.
\end{methoddesc}

\begin{methoddesc}{feed}{data}
Feeds \var{data} to the parser.
\end{methoddesc}

\begin{methoddesc}{get_parser_name}{}
Returns a single-word parser name.
\end{methoddesc}

\begin{methoddesc}{get_parser_version}{}
Returns the version of the imported parser, which may not be the
one the driver was implemented for.
\end{methoddesc}

\begin{methoddesc}{is_dtd_reading}{}
True if the parser is non-validating, but conforms to the XML specification by
reading the DTD.
\end{methoddesc}

\begin{methoddesc}{is_validating}{}
Returns true if the parser is validating, false otherwise.
\end{methoddesc}

\begin{methoddesc}{reset}{}
Makes the parser start parsing afresh.
\end{methoddesc}

\subsection{\class{ParserFactory} methods}

\begin{methoddesc}{get_parser_list}{}
Returns the list of possible drivers.  Currently this starts out as
\code{["xml.sax.drivers.drv_xmltok",
                      "xml.sax.drivers.drv_xmlproc",
                      "xml.sax.drivers.drv_xmltoolkit",
                      "xml.sax.drivers.drv_xmllib"]}.
\end{methoddesc}

\begin{methoddesc}{make_parser}{\optional{driver_name}}
Returns a SAX driver for the first available parser of the parsers
in the list. Note that the list contains drivers, so it first tries
the driver and if that exists imports it to see if the parser also
exists. If no parsers are available a \class{SAXException} is thrown.

Optionally, \var{driver_name} can be a string containing the name of
the driver to be used; the stored parser list will then not be used at
all.  
\end{methoddesc}

\begin{methoddesc}{set_parser_list}{list}
Sets the driver list to \var{list}.
\end{methoddesc}


\section{{xml.sax.saxlib}}

\begin{classdesc}{AttributeList}{}
Interface for an attribute list. This interface provides
    information about a list of attributes for an element (only
    specified or defaulted attributes will be reported). Note that the
    information returned by this object will be valid only during the
    scope of the \method{DocumentHandler.startElement} callback, and the
    attributes will not necessarily be provided in the order declared
    or specified.
\end{classdesc}

\begin{classdesc}{DocumentHandler}{}
Handle general document events. This is the main client
    interface for SAX: it contains callbacks for the most important
    document events, such as the start and end of elements. You need
    to create an object that implements this interface, and then
    register it with the \class{Parser}. If you do not want to implement
    the entire interface, you can derive a class from \class{HandlerBase},
    which implements the default functionality. You can find the
    location of any document event using the \class{Locator} interface
    supplied by \method{setDocumentLocator()}.
\end{classdesc}

\begin{classdesc}{DTDHandler}{}
Handle DTD events. This interface specifies only those DTD
    events required for basic parsing (unparsed entities and
    attributes). If you do not want to implement the entire interface,
    you can extend \class{HandlerBase}, which implements the default
    behaviour.
\end{classdesc}

\begin{classdesc}{EntityResolver}{}
This is the basic interface for resolving entities. If you create an object
    implementing this interface, then register the object with your
    \class{Parser} instance, the parser will call the method in your object to
    resolve all external entities. Note that \class{HandlerBase} implements
    this interface with the default behaviour.
\end{classdesc}

\begin{classdesc}{ErrorHandler}{}
This is the basic interface for SAX error handlers. If you create an object
    that implements this interface, then register the object with your
    Parser, the parser will call the methods in your object to report
    all warnings and errors. There are three levels of errors
    available: warnings, (possibly) recoverable errors, and
    unrecoverable errors. All methods take a SAXParseException as the
    only parameter.
\end{classdesc}

\begin{classdesc}{HandlerBase}{}
Default base class for handlers. This class implements the default
    behaviour for four SAX interfaces, inheriting from them all:
    \class{EntityResolver}, \class{DTDHandler},
    \class{DocumentHandler}, and \class{ErrorHandler}.  Rather than
    implementing those full interfaces, you may simply extend this
    class and override the methods that you need. Note that the use of
    this class is optional, since you are free to implement the
    interfaces directly if you wish.
\end{classdesc}

\begin{classdesc}{Locator}{}
Interface for associating a SAX event with a document
location. A locator object will return valid results only during
calls to methods of the \class{SAXDocumentHandler} class; 
at any other time, the results are unpredictable.
\end{classdesc}

\begin{classdesc}{Parser}{}
Basic interface for SAX parsers. All SAX
    parsers must implement this basic interface: it allows users to
    register handlers for different types of events and to initiate a
    parse from a URI, a character stream, or a byte stream. SAX
    parsers should also implement a zero-argument constructor.
\end{classdesc}


\begin{classdesc}{SAXException}{msg, exception, locator}
Encapsulate an XML error or warning. This class can contain
    basic error or warning information from either the XML parser or
    the application: you can subclass it to provide additional
    functionality, or to add localization. Note that although you will
    receive a \exception{SAXException} as the argument to the handlers in the
    \class{ErrorHandler} interface, you are not actually required to throw
    the exception; instead, you can simply read the information in
    it.
\end{classdesc}

\begin{classdesc}{SAXParseException}{msg, exception, locator}
Encapsulate an XML parse error or warning.
    
This exception will include information for locating the error in the
    original XML document. Note that although the application will
    receive a \exception{SAXParseException} as the argument to the
    handlers in the \class{ErrorHandler} interface, the application is not
    actually required to throw the exception; instead, it can simply
    read the information in it and take a different action.

Since this exception is a subclass of \exception{SAXException}, it inherits
    the ability to wrap another exception.
\end{classdesc}


\subsection{\class{AttributeList} methods}

The \class{AttributeList} class supports some of the behaviour of
Python dictionaries; the \method{len()}, \method{has_key()},
\method{keys()} methods are available, and \code{attr['href']} will
retrieve the value of the \attribute{href} attribute.  There are also
additional methods specific to \class{AttributeList}:

\begin{methoddesc}{getLength}{}
Return the number of attributes in the list.
\end{methoddesc}

\begin{methoddesc}{getName}{i}
Return the name of attribute \var{i} in the list.
\end{methoddesc}

\begin{methoddesc}{getType}{i}
Return the type of an attribute in the list. \var{i} can be
either the integer index or the attribute name.
\end{methoddesc}

\begin{methoddesc}{getValue}{i}
Return the value of an attribute in the list. \var{i} can be
either the integer index or the attribute name.
\end{methoddesc}

\subsection{\class{DocumentHandler} methods}

\begin{methoddesc}{characters}{ch, start, length}
Handle a character data event.
\end{methoddesc}

\begin{methoddesc}{endDocument}{}
Handle an event for the end of a document.
\end{methoddesc}

\begin{methoddesc}{endElement}{name}
Handle an event for the end of an element.
\end{methoddesc}

\begin{methoddesc}{ignorableWhitespace}{ch, start, length}
Handle an event for ignorable whitespace in element content.
\end{methoddesc}

\begin{methoddesc}{processingInstruction}{target, data}
Handle a processing instruction event.
\end{methoddesc}

\begin{methoddesc}{setDocumentLocator}{locator}
Receive 
an object for locating the origin of SAX document events.  You'll probably want to store the value of \var{locator} as an attribute of the handler instance.
\end{methoddesc}

\begin{methoddesc}{startDocument}{}
Handle an event for the beginning of a document.
\end{methoddesc}

\begin{methoddesc}{startElement}{name, attrs}
Handle an event for the beginning of an element.
\end{methoddesc}


\subsection{\class{DTDHandler} methods}

\begin{methoddesc}{notationDecl}{name, publicId, systemId}
Handle a notation declaration event.
\end{methoddesc}

\begin{methoddesc}{unparsedEntityDecl}{publicId, systemId, notationName}
Handle an unparsed entity declaration event.
\end{methoddesc}


\subsection{\class{EntityResolver} methods}

\begin{methoddesc}{resolveEntity}{name, publicId, systemId}
Resolve the system identifier of an entity.
\end{methoddesc}

\subsection{\class{ErrorHandler} methods}

\begin{methoddesc}{error}{exception}
Handle a recoverable error.
\end{methoddesc}

\begin{methoddesc}{fatalError}{exception}
Handle a non-recoverable error.
\end{methoddesc}

\begin{methoddesc}{warning}{exception}
Handle a warning.
\end{methoddesc}

\subsection{\class{Locator} methods}

\begin{methoddesc}{getColumnNumber}{}
Return the column number where the current event ends.
\end{methoddesc}

\begin{methoddesc}{getLineNumber}{}
Return the line number where the current event ends.
\end{methoddesc}

\begin{methoddesc}{getPublicId}{}
Return the public identifier for the current event.
\end{methoddesc}

\begin{methoddesc}{getSystemId}{}
Return the system identifier for the current event.
\end{methoddesc}

\subsection{\class{Parser} methods}

\begin{methoddesc}{parse}{systemId}
Parse an XML document from a system identifier.
\end{methoddesc}

\begin{methoddesc}{parseFile}{fileobj}
Parse an XML document from a file-like object.
\end{methoddesc}

\begin{methoddesc}{setDocumentHandler}{handler}
Register an object to receive basic document-related events.
\end{methoddesc}

\begin{methoddesc}{setDTDHandler}{handler}
Register an object to receive basic DTD-related events.
\end{methoddesc}

\begin{methoddesc}{setEntityResolver}{resolver}
Register an object to resolve external entities.
\end{methoddesc}

\begin{methoddesc}{setErrorHandler}{handler}
Register an object to receive error-message events.
\end{methoddesc}

\begin{methoddesc}{setLocale}{locale}
Allow an application to set the locale for errors and warnings. 
   
        SAX parsers are not required to provide localisation for errors
        and warnings; if they cannot support the requested locale,
        however, they must throw a SAX exception. Applications may
        request a locale change in the middle of a parse.
\end{methoddesc}

\subsection{\class{SAXException} methods}

\begin{methoddesc}{getException}{}
Return the embedded exception, if any.
\end{methoddesc}

\begin{methoddesc}{getMessage}{}
Return a message for this exception.
\end{methoddesc}

\subsection{\class{SAXParseException} methods}

The \class{SAXParseException} class has a \member{locator}
attribute, containing an instance of the \class{Locator} class, which
represents the location in the document where the parse error
occurred.  The following methods are delegated to this instance.

\begin{methoddesc}{getColumnNumber}{}
Return the column number of the end of the text where the exception
	occurred.
\end{methoddesc}

\begin{methoddesc}{getLineNumber}{}
Return the line number of the end of the text where the exception occurred.
\end{methoddesc}

\begin{methoddesc}{getPublicId}{}
Return the public identifier of the entity where the exception occurred.
\end{methoddesc}

\begin{methoddesc}{getSystemId}{}
Return the system identifier of the entity where the exception occurred.
\end{methoddesc}


\section{{xml.sax.saxutils}}

\begin{classdesc}{Canonizer}{writer}
A SAX document handler that produces canonicalized XML output.
\var{writer} must support a \method{write()} method which accepts a
single string.  
\end{classdesc}

\begin{classdesc}{ErrorPrinter}{}
A simple class that just prints error messages to standard error (\code{sys.stderr}).
\end{classdesc}

\begin{classdesc}{ESISDocHandler}{writer}
A SAX document handler that produces naive ESIS output.
\var{writer} must support a \method{write()} method which accepts a single string. 
\end{classdesc}

\begin{classdesc}{EventBroadcaster}{list}
Takes a list of objects and forwards any method calls received
    to all objects in the list. The attribute \member{list} holds the list and
    can freely be modified by clients.
\end{classdesc}

\begin{classdesc}{Location}{locator}
Represents a location in an XML entity. Initialized by being passed
a locator, from which it reads off the current location, which is then
stored internally.
\end{classdesc}

\subsection{\class{Location} methods}

\begin{methoddesc}{getColumnNumber}{}
Return the column number of the location.
\end{methoddesc}

\begin{methoddesc}{getLineNumber}{}
Return the line number of the location.
\end{methoddesc}

\begin{methoddesc}{getPublicId}{}
Return the public identifier for the location.
\end{methoddesc}

\begin{methoddesc}{getSystemId}{}
Return the system identifier for the location.
\end{methoddesc}

\section{\module{xml.utils.iso8601}
         %--- Utilities for handling ISO~8601 dates
         }

\declaremodule{}{xml.utils.iso8601}
\moduleauthor{Fred L. Drake, Jr.}{fdrake@acm.org}
\sectionauthor{Fred L. Drake, Jr.}{fdrake@acm.org}


The \module{xml.utils.iso8601} module provides conversion routines
between the ISO~8601 representations of date/time values and the
floating point values used elsewhere in Python.  The floating point
represtentation is particularly useful in conjunction with the
standard \module{time} module.

Currently, this module supports a small superset of the ISO~8601
profile described by the World Wide Web Consortium (W3C).  This is a
subset of ISO~8601, but covers the cases expected to be used most
often in the context of XML processing and Web applications.  Future
versions of this module may support a larger subset of
ISO~8601-defined formats.


\begin{funcdesc}{parse}{s}
Parse an ISO~8601 date representation (with an optional time-of-day
component) and return the date in seconds since the epoch.
\end{funcdesc}


\begin{funcdesc}{parse_timezone}{timezone}
Parse an ISO~8601 time zone designator and return the offset relative
to Universal Coordinated Time (UTC) in seconds.  If \var{timezone} is
not valid, \exception{ValueError} is raised.
\end{funcdesc}


\begin{funcdesc}{tostring}{t\optional{, timezone}}
Return formatted date/time value according to the profile described by 
the W3C.  If \var{timezone} is provided, it must be the offset from
UTC in seconds specified as a number, or time zone designator which
can be parsed by \function{parse_timezone()}.  If \var{timezone} is
specified as a string and cannot be parsed by
\function{parse_timezone()}, \exception{ValueError} will be raised.
\end{funcdesc}


\begin{funcdesc}{ctime}{t}
Return formatter date/time value using the local timezone.  This is
equivalent to \samp{tostring(\var{t}, time.timezone)}.
\end{funcdesc}


\begin{seealso}
\seetext{International Organization for Standardization.  \emph{Data
elements and interchange formats --- Information interchange ---
Representation of dates and times.}  International Organization for
Standardization, 1988.}

\seetext{Gary Houston.  \emph{ISO~8601 date/time representations}.
January 1993.  Available online as compressed PostScript:
\url{ftp://ftp.informatik.uni-erlangen.de/pub/doc/ISO/ISO8601.ps.Z}.}

\seetext{Markus Kuhn.  \emph{A Summary of the International Standard
Dateand Time Notation}.  Available online at
\url{http://www.cl.cam.ac.uk/~mgk25/iso-time.html}.}

\seetext{Misha Wolf and Charles Wicksteed.  \emph{Date and Time
Formats}.  World Wide Web Consortium Technical Note, September 1998.
Available online at \url{http://www.w3.org/TR/NOTE-datetime}.}
\end{seealso}


\end{document}
